\documentclass[a4paper,11pt]{jsarticle}
\usepackage[dvipdfmx]{graphicx}
\usepackage[utf8]{inputenc}
\usepackage{url}


\title{Webプログラミング 期末レポート 仕様書}
\author{25g1146 渡邊煌基}
\date{\today}

\begin{document}

\maketitle
\begin{center}
    \url{https://github.com/kirameki83-dot/webpro_06}
\end{center}

\section{はじめに}
本課題では,共通の操作感(一覧表示,詳細表示,追加,変更,削除)を持つ3つのWebアプリケーションを作成した.
全てのアプリケーションはデータベースを使用せず,変数内でデータを管理する実装を行っている.
また,ソースコードは Github (\url{https://github.com/kirameki83-dot/webpro_06}) にて公開している.

\section{開発したアプリケーション}
\subsection{IIDX Score Manager}
音楽ゲーム「beatmania IIDX」のスコアを管理するツール.
ユーザーはプレイした楽曲のスコア,難易度,クリアランプ等を記録できる.

\subsection{Course Manager}
大学の授業科目を管理するツール.
科目名,担当教員,開講期(前期・後期など),時限,単位数,区分(必修・選択など)を登録し,
履修計画の確認や単位数の計算(目視確認)に役立てる.

\subsection{To-Do List}
日常のタスクを管理するシンプルなアプリケーション.
タスク名,期限,優先度,状態(未完了・完了)を管理し,すべきことを整理できる.

\section{利用者向けマニュアル}
\subsection{共通操作}
全てのアプリケーションで以下の操作は共通している.
\begin{itemize}
    \item \textbf{一覧表示}: アプリケーションのトップページにアクセスすると,登録されているデータのリストが表示される.
    \item \textbf{詳細表示}: リスト内の「詳細」リンクをクリックすると,そのデータの詳細情報が表示される.
    \item \textbf{新規作成}: 一覧画面の「新規登録/作成」ボタンから,新しいデータを登録できる.
    \item \textbf{編集}: 詳細画面の「編集」ボタンから,既存データを修正できる.
    \item \textbf{削除}: 詳細画面の「削除」ボタンをクリックすると,確認ダイアログが表示され,OKを押すとデータが削除される.
\end{itemize}

\subsection{各アプリ固有の項目}
\begin{itemize}
    \item \textbf{IIDX Score Manager}: バージョン,難易度,DJレベル,ランプなどはプルダウンから選択する形式となっている.
    \item \textbf{Course Manager}: 開講期や区分は大学のカリキュラムに合わせて選択する.
    \item \textbf{To-Do List}: 優先度によって一覧表示時の左側の色が変化する(高:赤,中:黄,低:緑).完了状態にすると取り消し線が表示される.
\end{itemize}

\section{管理者向けマニュアル}
本システムは Node.js 環境で動作する.

\subsection{デプロイ手順}
\begin{enumerate}
    \item ソースコード一式をサーバーに配置する.
    \item コンソールでディレクトリに移動し,\texttt{npm install} を実行して依存ライブラリをインストールする.
    \item 各アプリケーションの起動コマンドを実行する.
    \begin{itemize}
        \item IIDX: \texttt{node app\_iidx.js} (Port: 8081)
        \item Course: \texttt{node app\_course.js} (Port: 8082)
        \item ToDo: \texttt{node app\_todo.js} (Port: 8083)
    \end{itemize}
    \item ブラウザで \texttt{http://localhost:[ポート番号]} にアクセスして動作を確認する.
\end{enumerate}

\section{開発者向けマニュアル}
\subsection{技術スタック}
\begin{itemize}
    \item \textbf{Node.js}: サーバーサイド実行環境.
    \item \textbf{Express}: Webアプリケーションフレームワーク.ルーティングやミドルウェアの利用を容易にするために採用.
    \item \textbf{EJS}: テンプレートエンジン.HTML内にJavaScriptを埋め込み,動的なページ生成を行うために採用.
\end{itemize}

\subsection{実装の詳細}
\subsubsection{データ管理}
課題要件に従い,データベースは使用せず,グローバル変数(配列)にてデータを管理している.
サーバーを再起動するとデータは初期化される仕様である.
各データには一意のIDが付与され,CRUD操作はこのIDを用いて対象を特定する.

\subsubsection{Strict Mode}
全てのJSファイルの先頭に \texttt{"use strict";} を記述している.
これにより,未宣言変数の使用や,予約語の使用などをエラーとして検出し,より安全なコード記述を強制している.

\subsubsection{スタイル}
CSSは各テンプレートファイル内に記述,または共通のスタイル定義を用いている.
それぞれのアプリケーションのテーマカラー(IIDX:青/黒, Course:緑/白, ToDo:紫/白)を設定し,視認性を高めている.

\end{document}
